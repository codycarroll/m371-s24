\documentclass{exam}
\usepackage{url,graphicx,tabularx,array,geometry,enumitem,amsmath,
color,fullpage,datetime,caption,subcaption,float,mathtools}
\begin{document}

\def\bco{\iffalse}
\newcommand{\bea}{\begin{eqnarray*}}
\newcommand{\eea}{\end{eqnarray*}}
\newcommand{\be}{\begin{eqnarray}}
\newcommand{\ee}{\end{eqnarray}}
\newcommand{\bsp}{\begin{split}}
\newcommand{\esp}{\end{split}}
\newcommand{\ed}{\end{document}}
\newcommand{\no}{\noindent}
\newcommand{\ben}{\begin{enumerate}}
\newcommand{\een}{\end{enumerate}}
\newcommand{\bena}{\begin{enumerate}[label = \alph*)]}

\centerline{\bf \sc \LARGE Exam 1 - Practice Questions}
\vspace{.5pc}

\vspace{1cm}

\textbf{Rules:} 

\begin{itemize}

%\item You may start the quiz anytime between 9pm on Wednesday to 8pm Thursday (Pacific Standard Time) on the announced Wednesday date that varies for each quiz. 



\item Show all your work and reasoning how you arrive at your answer. Just writing an answer is not sufficient for most problems. 

\item When in doubt, \textit{draw the picture!!!}

\end{itemize} 


\bco


$$ \begin{aligned}
\textbf{Name}&: \rule{5cm}{0.15mm} \\\\
\textbf{ID}&: \rule{5cm}{0.15mm}
\end{aligned} $$

\fi

\vspace{.1in}


\textbf{Problems:} 

\begin{enumerate}

%\item Suppose that $X_1, \dots, X_n$ form a random sample from a distribution for which the p.d.f. $f(x|\theta)$ is as follows:
%$$ f(x|\theta) = \begin{cases}
%						\theta x^{\theta - 1} &\mbox{for } 0 < x < 1, \\
%						0 &\mbox{otherwise. }
%					  \end{cases} $$
%Also, suppose the value of $\theta$ is unknown,  and satisfies  $\theta>0$.
%\begin{enumerate}[label = \alph*)]
%\item Determine the joint p.d.f. of $X_1, \dots, X_n$.
%\item Find the MLE of $\theta$.
%\item  Find the MLE of $\theta^2$.
%\end{enumerate}


\item Let $X_1, \dots, X_n$ be a random sample from a uniform distribution U$[-1,\theta]$, where $\theta > -1.$
\begin{enumerate}[label = \alph*)]
\item Obtain the MLE for $EX_1$.
\item Obtain the MLE for var$(X_1)$.
\item Obtain two Method of Moments estimators for var$(X_1)$.
\item Now suppose we know that $\theta \ge 0.$ Obtain the MLE for $\theta$.

\een 

%\item In a study we record the number of Covid-19 infections for each of 100 counties on the West Coast of the United States on Jan. 20, 2021.  
%Assume that these numbers are iid and follow a Poisson distribution with parameter $\lambda$. 
%
%\bena
%
%\item Write the likelihood function (not log likelihood) for $\lambda$. Derive it step by step and  provide the arguments that are used in its derivation.   
%\item Obtain two Method of Moments estimators for the standard deviation of this distribution. 
%\item Obtain the MLE for the standard deviation and compare it with the two Method of Moments estimators. 
%
%\een



\item You participate in a study that aims to develop a new vaccine to protect against Covid-19 infection. The new vaccine has been applied in two independent trials. The first 
trial has led to a sample of size $n$  from a Bernoulli$(\theta_1)$ model and the second trial to a sample of size $m$ from a Bernoulli$(\theta_2)$ model, where $\theta_1, \theta_2$ are the success probabilities that the vaccine is effective. 

\bena

\item Obtain the MLEs $\hat\theta_1, \hat\theta_2$ for $\theta_1, \theta_2$.
\item Provide arguments why $\theta_1, \theta_2$ might be the same or might be different. 
What about $\hat\theta_1, \hat\theta_2$?
\item It is now assumed that  $\theta_1=\theta_2$. Calculate 
var$(\hat\theta_1)$, var$(\hat\theta_2)$.
\item Under this assumption, a scientist suggests to combine the two MLEs in a suitable way to obtain an improved estimator that takes all of the available information into account.  
For that, one considers combinations $\hat{\theta}_c=c \hat\theta_1 + (1-c) \hat\theta_2$ for $0 \le c \le 1.$
Find the best $c$ such that var$(\hat{\theta}_c)$  is minimized. 

\een



\item  Consider a sample of size $n$ from an exponential distribution with parameter $\beta>0.$  Assume the r.v. $X$ has this distribution. 

\bena
\item Find a method of moments estimator for $\beta.$
\item Find a method of moments estimator for $P(0 \le X \le 2).$
\item Find a method of moments estimator for the median of the distribution. 

\een
\een
\ed




\end{enumerate}


\end{enumerate}


\end{document}